\documentclass{report}
\usepackage[utf8]{inputenc}
\usepackage{color}
\usepackage{listings} % needed for the inclusion of source code
\usepackage{mips}

\definecolor{dkgreen}{rgb}{0,0.6,0}
\definecolor{gray}{rgb}{0.5,0.5,0.5}
\definecolor{mauve}{rgb}{0.58,0,0.82}

\lstset{ %
  language=[mips]Assembler,       % the language of the code
  basicstyle=\footnotesize,       % the size of the fonts that are used for the code
  numbers=left,                   % where to put the line-numbers
  numberstyle=\tiny\color{gray},  % the style that is used for the line-numbers
  stepnumber=1,                   % the step between two line-numbers. If it's 1, each line 
                                  % will be numbered
  numbersep=5pt,                  % how far the line-numbers are from the code
  backgroundcolor=\color{white},  % choose the background color. You must add \usepackage{color}
  showspaces=false,               % show spaces adding particular underscores
  showstringspaces=false,         % underline spaces within strings
  showtabs=false,                 % show tabs within strings adding particular underscores
  frame=single,                   % adds a frame around the code
  rulecolor=\color{black},        % if not set, the frame-color may be changed on line-breaks within not-black text (e.g. commens (green here))
  tabsize=4,                      % sets default tabsize to 2 spaces
  captionpos=b,                   % sets the caption-position to bottom
  breaklines=true,                % sets automatic line breaking
  breakatwhitespace=false,        % sets if automatic breaks should only happen at whitespace
  title=\lstname,                 % show the filename of files included with \lstinputlisting;
                                  % also try caption instead of title
  keywordstyle=\color{blue},          % keyword style
  commentstyle=\color{dkgreen},       % comment style
  stringstyle=\color{mauve},         % string literal style
  escapeinside={\%*}{*)},            % if you want to add a comment within your code
  morekeywords={*,...}               % if you want to add more keywords to the set
}

\title{Computer Fundamentals Diary}
\author{Arroyo Calle, Adrián \\ Alonso Pascual, Sergio}
\date{March 2017}

\begin{document}

\maketitle

\tableofcontents

\chapter{March, Week of 13th}

\section{Thursday}

That was our first encounter with the MARS simulator for the MIPS processor.

\subsection{Analysis}

We analyzed the code given to us. It couldn't compile, so we fix that. We also realized that the code compiled without commas in some instructions. The program sums the elements of a vector and puts the result in \textbf{\$s3}.

\subsection{f = f - (g+h-2) }

Next, we build the program for that operation, holding f,g and h in \$s0, \$s1, \$s2 with the following initial values: 1, 10 and 5. We tried to use the minimum number of registers.

\lstinputlisting{MIPS2.s}

\subsection{Same but with memory}

Next, we wrote a program that did the same thing as the previous one but using data from memory. We used one more register than in the previous version. Or we could just use three registers with more instructions (doing la of F twice).

\lstinputlisting{MIPS3.s}

\subsection{f = 8*(g+h)+4*(f+h)}

Next, we wrote two programs for that expression. We realized that sometimes we can just use lw with the name of a address as expressed in .data. This reduces the number of instructions by two, because la it's a pseudoinstruction made up with lui and ori.

\section{Friday}

\subsection{Deliverable}

For the deliverable we started with a basic program that just did what it was requested.

\lstinputlisting{MIPS7.s}

Then we realized we could just compare with \$zero in the bne instruction. In order to do that, we did a reverse loop, using one less register (in the initialization section) and one less instruction.

\lstinputlisting{MIPS7-fast.s}

\chapter{March, Week of 20th}

\section{Thursday}
Today we wrote the programs for the first exercise. At first we did while loops like they were do-while statements, but we later realized that it was wrong and we changed it.
We continued doing the rest exercises without any major trouble until we got to the deliverable.
At first we tried to copy the sentence without spaces and we succeeded, then we tried to reverse the word by copying it from the end of B but it left space between the two words, then we tried subtracting A from B to remove that space but the teacher told us that it was wrong because A and B don`t have to be adjacent. As the hour was ending we decided to solve it the next day.
\section{Friday}
We started the deliverable from scratch with a new approach. We read A from its ending and copy it to B, except if it is a space. It worked.

\chapter{March, Week of 27th}

\section{Thursday}
Today we started with the third assignment. It was about procedures and functions. We tried not to use \$s0-7 registers because we already knew that it was painful to use them in a function. 

The first program was easy. The second program was a bit more complicated. We thought that 0 is a natural even number. Next, we split the program to use a function. It wasn't difficult but we needed to change some registers.

\lstinputlisting{MIPS23.s}

For the fourth task we looked at some older files and adapted them to use a function.

For the deliverable we started with a program that converts strings to uppercase. We already knew that in ASCII, uppers and lowers characters are separated by the same number (32). We checked if the word is already upper or not (less than 91). If it is, we leave it and save it. If it isn't, we substract 32 in the character.
\end{document}

